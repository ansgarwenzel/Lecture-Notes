%% Based on a TeXnicCenter-Template by Gyorgy SZEIDL.
%%%%%%%%%%%%%%%%%%%%%%%%%%%%%%%%%%%%%%%%%%%%%%%%%%%%%%%%%%%%%

%----------------------------------------------------------
%
\documentclass[a4paper,12pt,reqno]{amsbook}%
%
%----------------------------------------------------------
% This is a sample document for the AMS LaTeX Book or Monograph Class
% Class options
%       --  Body text point size:
%                        8pt, 9pt, 10pt (default), 11pt, 12pt
%       --  Paper size:  letterpaper (8.5x11 inch, default), a4paper
%       --  Orientation: portrait(default), landscape
%       --  Print side:  oneside, twoside (default)
%       --  Quality:     final(default), draft
%       --  Title page:  titlepage, notitlepage
%       --  Start chapter on left:
%                        openright (no, default), openany
%       --  Columns:     onecolumn (default), twocolumn
%       --  Omit extra math features:
%                        nomath
%       --  AMS fonts (noamasfonts available):
%                        noamsfonts
%       --  PSAMSfonts (fewer AMSfontsizes)
%                        psamsfonts
%       --  Equation numbering (equation numbers on the left is the default)
%                        leqno (default), reqno
%       --  Equation centering (equations centered is the default)
%                        centeredtags (default}, tbtags (top, bottom)
%       --  Displayed equations (centered is the default)
%                        fleqn (flush left)
% For instance the command
%          \documentclass[a4paper,12p,reqno]{amsbook}
% ensures that the paper size is a4, fonts are typeset at the size 12p
% and the equation numbers are on the right side.
%
\usepackage{amsmath}%
\usepackage{amsfonts}%
\usepackage{amssymb}%
\usepackage{graphicx}
%----------------------------------------------------------
\theoremstyle{plain}
\newtheorem{acknowledgement}{Acknowledgement}
\newtheorem{algorithm}{Algorithm}
\newtheorem{axiom}{Axiom}
\newtheorem{case}{Case}
\newtheorem{claim}{Claim}
\newtheorem{conclusion}{Conclusion}
\newtheorem{condition}{Condition}
\newtheorem{conjecture}{Conjecture}
\newtheorem{corollary}{Corollary}
\newtheorem{criterion}{Criterion}
\newtheorem{definition}{Definition}
\newtheorem{example}{Example}
\newtheorem{exercise}{Exercise}
\newtheorem{lemma}{Lemma}
\newtheorem{notation}{Notation}
\newtheorem{problem}{Problem}
\newtheorem{proposition}{Proposition}
\newtheorem{remark}{Remark}
\newtheorem{solution}{Solution}
\newtheorem{summary}{Summary}
\newtheorem{theorem}{Theorem}
\numberwithin{equation}{section}
%-----------------------------------------------------------
\begin{document}
\frontmatter
\title[Group Representation Theory]{A (very) short introduction to Group Representation Theory}
\author[A. Wenzel]{Ansgar Wenzel}
\address[]{Chichester, 3R400%
\hspace*{\fill}\linebreak\indent%
University of Sussex}%
%\curraddr[P. Carlson]{Author One, current address, line 1
%\hspace*{\fill}\linebreak
%\indent Author One, current address, line 2}%
\email[]{a.wenzel@sussex.ac.uk}
%\urladdr{http://www.authorone.uni-aone.de}
%\author{Name of the Second Author}
%\thanks{The Author thanks J. Smith}
%\subjclass{Primary 05C38, 15A15; Secondary 05A15, 15A18}
%\dedicatory{Dedicated to Professor XY}
\include{abstract}
\maketitle
\tableofcontents


\chapter*{Introduction}

\markboth{PREFACE}{PREFACE} 
This comprises the lecture notes for the second part of the course \emph{Group Theory II}.
It is based mainly on~\cite{Sagan}, in particular Chapters 1 and 2, as well as lecture notes of a course the author took a few years ago.
\mainmatter

\part{Group Representations}

\chapter{Group Theory Prerequisites}

In this chapter, we are introducing the basic group theoretical notions required for this course.
We are going to focus on the Symmetric Group in particular.
This chapter is based on~\cite{Grillet}, in particular chapter I.1 and II.4.
We will assume familiarity with the basic group theoretical definitions.
As a reminder, $S_n$ consists of all bijections from $\mathbb{Z}_n$ to itself, using composition as the group operation.
Throughout the course, G is written multiplicatively with identity $\mathfrak{e}$.
\begin{definition}
Let $a,b\in\mathbb{Z}_n$, $a\neq b$.
The \emph{transposition} $\tau=(a\;b)$ is the permutation defined by $\tau a=b,\;\tau b=a$ and $\tau x=x$ for all $x\neq a,b$.
\end{definition}
\begin{proposition}
Every permutation $\pi\in S_n$ is the product of transpositions.
\end{proposition}
The proof is not important for the course and left as an exercise.
\begin{exercise}
Show that $(1;2),(2\;3),\ldots,(n-1\;n)$ form a basis and, indeed, generate $S_n$.
\end{exercise}
\footnote{could include Braid groups? maybe as an alternative to $\pi_0$? Certainly easier to do questions for the exam with.}
If $\pi$ is a permutation, there are three different notations we can use.
The \emph{two line notation} is probably the most common and is as follows:
\begin{equation*}
\pi=
\begin{array}{cccc}
1&2&\ldots &n\\
\pi(1)&\pi(2)&\ldots&\pi(n)
\end{array}
\end{equation*}
In order to get the \emph{one line notation}, we drop the first line, as it is fixed.
We can also display $\pi$ using \emph{cycle notation}.

\backmatter \appendix

\chapter{The First Appendix}


\chapter{The Second Appendix}


\begin{thebibliography}{9}
\bibitem{Sagan} \textsc{Bruce E. Sagan}:\ \textit{The Symmetric Group - Representations, Combinatorial Algorithms and Symmetric Functions}, \textbf{Springer Verlag}, New York, 2001.
\bibitem{Grillet} \textsc{Pierre Antoine Grillet}:\ \textit{Abstract Algebra}, \textbf{Springer Science and Media}, New York, 2007.
\end{thebibliography}

\end{document}
